
\chapter{Introduction}

% what is the problem ?
Speech recognition is the act of correctly assigning lexical items to an acoustic speech signal.
Doing so in an automated way has proven to be a considerable challenge. Although the task seems somehow trivial as we communicate effortlessly with others around us, there are many difficulties arising when one attempts to build a computational model as efficient as we are.

First, each spoken word is composed of speech sounds that are influenced by one another. 
% TODO example of coarticulation
For example

Secondly, spoken words in a sentence are not usually separated by silence. This means that word boundaries have to be guessed by the model - a problem a written text recogniser does not have to deal with, as written words are separated by blanks.
%TODO cite (taken from McClelland86) :
% there are some cues in the speech stream, but as several investigators have pointed out (Cole & Jakimik, 1980; Grosjean & Gee, 1984; Thompson, 1984), they are not always sufficient,
Finally, we often communicate in noisy environments, and each speaker has its own accent and way of pronouncing, not to mention that the intonation of the same utterance by a single speaker changes depending on his mood.

All these facts are a reason as to why no computational model has managed to match our ability to recognise speech. 

% why is it important ?

An approach to solving the problem of automatic speech recognition is the development of computational cognitive models, in which much attention is draw to the cognitive processes involved in speech recognition. These models aim at reproducing empirical data collected through diverse experiments made on human subjects. 


% talk about TRACE

% briefly mention the abstract / concrete universal stuff
% present what the paper is about


\section{Overview}
% overview of the rest of the paper

